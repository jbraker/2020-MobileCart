\section{Objectives}
By the end of this laboratory assignment, students are expected to learn how to 

\begin{itemize}

\item interface an analog sensor with an analog-to-digital converter device

\item use an embedded single-board computer that reads analog signals fed by external analog  sensors and   
  
\item utilize API functions for displaying sensor outputs (voltages) that are fed from an  external analog sensor.  
  
\end{itemize}

\section{Parts}
\label{sec:partsADC}
A list of parts needed for conducting  this laboratory assignment is given below. %
%
\begin{enumerate}
\item BBBlue board with the latest OS (debian) image installed
\item One micro-USB cable
  
\item Breadboard
  
\item One LM34 precision fahrenheit temperature sensor\footnote{See the datasheet at \href{http://www.ti.com/lit/ds/symlink/lm34.pdf}{http://www.ti.com/lit/ds/symlink/lm34.pdf}}
  
\item One 6-pin JST SH (1mm pitch) connector, with 150mm 28AWG wires. It can be used to work with ADC, GPIO, GPS, and SPI.

  
\item One LMC6482\footnote{See the datasheet at \href{http://www.ti.com/lit/ds/symlink/lmc6482.pdf}{http://www.ti.com/lit/ds/symlink/lmc6482.pdf}} dual operational amplifier IC

  
\item Four resistors 
\end{enumerate}

\section{Background}
\label{sec:backgroundADC}

Time-varying analog signals (usually voltages or currents) generated by sensors, such as temperature sensors, are to be converted to digital signals using analog-to-digital converters (ADC) for further processing. For that, analog signals are sampled at a constant time interval, $T_s.$ Therefore, the rate at which analog signals are sampled is called the \emph{sampling frequency} $f_s=1/T_s.$   The sampling frequency of an analog signals affects its discrete-time representation~\cite[Ch.~10]{Smaili2008}. To preserve frequency components of the original analog signal, sampling frequency, $f_s,$ must be more than twice the maximum frequency of the original signal, \textit{i.e.,~} %
%
%\begin{align*}
  $f_s > 2f_{\text{max}},$ 
%\end{align*}
%
where $f_{\text{max}}$ is the highest frequency component of the original analog signal. Note that the frequency $2f_{\text{max}}$ is called the \emph{Nyquist frequency}. Digitized version of the original signal is misrepresented by ADC as a \emph{low-frequency signal}. This phenomenon is called \emph{aliasing}.  The detailed theoretical background on sampling and analog--to--digital conversion techniques of analog signals are omitted here for conciseness.   

In this laboratory, an analog-to-digital converter (ADC) embedded into a single-board computer, BBBlue, will be used to read analog signals (voltages) produced by a temperature sensor. The temperature sensor used in this laboratory assignment produces an analog output voltage that is proportional to the ambient temperature. BBBlue has a 12-bit ADC with a full scale of $\text{FS} = V_{RH} - V_{RL} = 1.8-0=1.8~[\volt],$ where $V_{RH}$ and $V_{RL}$ are the high and low reference voltages of the ADC, respectively. There are four ADC input channels in the socket shown in Figure~\ref{fig:BBBlue}. Four ADC input channels  are accessible via a 6-pin JST-SH connector. 
%

In general, there are four potentiometers that can be connected with the ADC socket for reading analog signals. The pinout of the socket is as follows: %
%
\begin{center}
  \begin{tabular}{l}
    PIN 1 -- Ground\\
    PIN 2 -- VDD\_ADC $(1.8~[V])$\\
    PIN 3 -- AIN0\\
    PIN 4 -- AIN1\\
    PIN 5 -- AIN2\\
    PIN 6 -- AIN3  
  \end{tabular}
\end{center}
%
The ADC samples its input analog signals every $125~[\nano\second].$ Let $v_s$ denote the output voltage  of a temperature sensor and the input analog signal that is fed into one of  the ADC's input channels is denoted by $v_i.$  Since the ADC's high and low references voltages are $V_{RH} = 1.8~[\volt]$ and $V_{RL} = 0~[\volt],$ the input voltage to the ADC's input channels should be in this range,~\textit{i.e.~} $v_i\in[0,1.8]~\volt.$ Therefore, an input  signal conditioning circuit is required convert the output voltage, $v_s,$ of the temperature sensor into the range of the ADC. This is illustrated in the block diagram shown in  Figure~\ref{fig:BBBlue-ADC-BD}, where $(N)_2$ is the binary number that corresponds to the input voltage $v_i.$ %
%
\begin{figure}
  \centering
              \begin{tikzpicture}
                % \tikzstyle{every node} = [font =\tiny]
                % \tikzstyle{block} = [draw, fill=blue!20, rectangle, 
                % minimum height=3em, minimum width=6em]
                \tikzstyle{block} = [draw, fill=blue!20, rectangle, rounded corners]      
                \tikzstyle{pinstyle} = [pin edge={to-,thin,black}]
                % 
                % 
                \node [block,text width=2.0cm] (tempSensor) {Temperature sensor};
                \node[block, text width=2.0cm, right of =tempSensor, node distance=3.5cm](inSignalConditioning){Input signal conditioning};
                \node[block, text width=2.0cm, right of =inSignalConditioning, node distance=4.5cm](ADC){BBBlue's 12-bit ADC};                
                % Draw arrows
                \draw[->] (tempSensor) -- node[midway,above]{$v_s$}  (inSignalConditioning);
                \draw[->] (inSignalConditioning) -- node[midway,above]{$v_i\in[0,1.8]~\volt$}  (ADC);
                \draw[->](ADC) -- ++(3*\smgrid,0)node[right]{$(N)_2$};
              \end{tikzpicture}  
  \caption{Block diagram of reading the output of a temperature sensor using BBBlue's ADC.}
  \label{fig:BBBlue-ADC-BD}
\end{figure}
%
The input signal conditioning circuit can be implemented using inverting amplifiers as shown in Figure~\ref{fig:inputSignalConditioning}. 
%
\begin{figure}
  \centering
  \begin{circuitikz}[scale=1,american voltages]
    \draw
    (0,0) node[op amp,fill=cyan!50] (opamp1){}
    (-8*\smgrid,\smgrid) node[left]{$v_s$} to[R,l^=$R_1$,o-*](opamp1.-)
    (opamp1.+)to[short,-*](opamp1.+)node[ground]{};
    \node at(0,0){\#1};
    \draw
    (opamp1.-) to[short,-](-2.4*\smgrid,4*\smgrid) to[R,l^=$R_f$](2.4*\smgrid,4*\smgrid) to[short,-*](opamp1.out);
    
    \draw
    (10*\smgrid,-\smgrid) node[op amp,fill=cyan!50](opamp2){}
    (opamp1.out) to[R,l^=$R$,-*](opamp2.-)
    (opamp2.-) to[short,-](7.5*\smgrid,4*\smgrid) to[R,l^=$R$](12.4*\smgrid,4*\smgrid) to[short,-*](opamp2.out);
    \node at(10*\smgrid,-\smgrid){\#2};
    \draw 
    (opamp2.+) to[short,-*](opamp2.+)node[ground]{};
    
    \draw
    (opamp2.out) to[short,-o](14*\smgrid,-\smgrid) node[right]{$v_i$};
    
  \end{circuitikz}
  \caption{A possible input signal conditioning circuit using op-amps.}
  \label{fig:inputSignalConditioning}
\end{figure}
%



Note that the input voltage $v_i \in [V_{RL},V_{RH}]$ is converted  into a decimal number $(N)_{10}$ using a $k$-bit ADC, which is computed by: %
%
\begin{equation*}
(N)_{10} = \left\{\text{Round}\left(\frac{v_i-V_{RL}}{V_Q}\right)\right\}_{\text{truncate to $(2^k-1)$}},
\end{equation*}
%
where $  V_Q = \frac{\text{FS}}{2^k}$ is the resolution of the ADC. To approximate input voltage $v_i$ that corresponds to specific output $(N)_{10}$, the following relation is used:
    \begin{equation*}
      \left(v_i\right)_{\text{est.}} = (N)_{10}\times V_Q + V_{RL}.
    \end{equation*}
    %
    The input voltage $v_i$ at channel $\mathrm{AN}i,~0\le i\le 3,$ can be read using C library functions of the robotics cape package, \emph{librobotcontrol}, which is  pre-installed in the BBBlue board. The APIs used to read analog input voltage $v_i$ are: %
%
    
    \begin{center}
      \begin{tabular}{l|p{3.5in}}
        \toprule
        API & Description\\
        \toprule 
        rc\_adc\_read\_raw(int ch) & returns the raw integer output of the 12-bit ADC\\
        rc\_adc\_read\_volt(int ch) & converts the  raw value to a voltage\\
        \bottomrule
      \end{tabular}
\end{center}
%
See \url{http://strawsondesign.com/docs/librobotcontrol/group___a_d_c.html} for additional details of several APIs that are related to the BBBlue's ADC operations.   For convenience, a example program that prints voltages read by all ADC channels of the ADC is provided in Appendix~\ref{sec:appRC-TestADC}. 



\begin{prelab}[Sampling frequency]
  
Using Matlab, plot the time-varying analog signal given by the function  $v(t) = sin(t) + sin(2t),$ for time $t\in [0,10]~\second,$ at sampling frequencies corresponding to
\begin{enumerate}
\item[a.] 1/3,
\item[b.] 2/3,
\item[c.] 2,~ and
\item[d.] 10
\end{enumerate}
times the Nyquist frequency. Present the results in four separate figures. On each figure plot the sampled signal (solid line) along with the original one (dashed line). Comment on the results.  [Hint: To plot the original analog signal, $v(t),$ use a very small sampling time, $t_s = 0.0001~[\second],$ for instance.]
\end{prelab}

\begin{prelab}[Interfacing ADC with a  temperature sensor]{prelab:tmpSensorADC}

  
A temperature sensor with a gain of $10~[\milli\volt/\degreeCelsius]$ is used to measure the temperature of a process within the range of $-50~[\degreeCelsius]$ to $+200~[\degreeCelsius].$ A $12$-bit ADC with a range from $0~[\volt]$ to $+1.8~[\volt]$ is used. A signal-conditioning circuit is needed to match the limits of the sensor output $v_s$ with the input voltage $v_i$ of the ADC. 

\begin{itemize}
\item[a.] Determine a linear relationship for the signal-conditioning circuit between $v_s$ and $v_i$.
\item[b.] Determine the offset, full scale, and resolution of the measurement in terms of voltage and temperature.
\item[c.] Determine the output of the A/D conversion if the temperature is $+50~[\degreeCelsius]$, and express it in decimal, binary, and hex equivalents. 
\end{itemize}
\end{prelab}




\section{Laboratory Work}
%
Before you start working on this laboratory experiment, it is highly recommended that you go through the example program shown in Appendix~\ref{sec:appRC-TestADC} and run it on the BBBlue board. In addition, check the datasheet of the Op-Amp IC and implement a simple voltage follower circuit shown in Figure~\ref{fig:voltageFollowerCircuit1}.  The output of the circuit shown in Figure~\ref{fig:voltageFollowerCircuit1}, $v_{\text{out}}(t),$ should be the same as its input signal $v_{\text{in}}(t) = 5\sin(4000\pi t),$ for time $t\ge 0,$ for instance.  Note that a voltage follower circuit, such as the one shown in Figure~\ref{fig:voltageFollowerCircuit1}, has a voltage gain of unity (\textit{i.e.,~} $A_v=1).$
%
\begin{figure}
  \centering
  \begin{circuitikz}[scale=1, american voltages]
    \draw 
    (5*\smgrid,-0.5)node[op amp,yscale=-1,fill=cyan!50](opamp1){};
    \draw 
    (0,-5*\smgrid) to[sV,v<=$v_{\text{in}}(t){=}5\sin(4000\pi t)$,fill=green!50] (0,0) --(opamp1.+)
    (opamp1.out) to[short,-o](12*\smgrid,-0.5) to [open,v>=$v_{\text{out}}(t)$] (12*\smgrid,-5*\smgrid) to[short,o-*] (0,-5*\smgrid)node[ground]{};
    \draw
    (opamp1.-) -- (2.65*\smgrid,-4*\smgrid) -|(opamp1.out) to[short,-*](opamp1.out);
  \end{circuitikz}
  \caption{Voltage follower with a unity voltage gain (\textit{i.e.,} $A_v = 1$).}
  \label{fig:voltageFollowerCircuit1}
\end{figure}
%



\begin{enumerate}
\item Read the datasheet of the temperature sensor and determine its range of output voltages $v_s.$ 

  
\item Design and construct the signal conditioning circuit shown in Figure~\ref{fig:inputSignalConditioning} based  on the input signal conditional circuit  analysis performed in your prelab. [Note: New signal conditioning circuit parameters could be different from the prelab analysis.] 

 
\item Take three values (voltages) that correspond to three different temperature values and record the corresponding output voltages, $v_i,$ of the signal conditioning circuit.  

\item Connect the output of the signal conditioning circuit to $\mathrm{AN}0$ channel of the ADC socket of the BBBlue. 
  
\item Write C code for the BBBlue that reads three different voltages corresponding to three temperature values. 
   
\end{enumerate}


%%% Local Variables:
%%% mode: latex
%%% TeX-master: "../../labHandoutMechatronics-V1"
%%% End:

