\section{Objectives}
By the end of this laboratory experiment, students will learn how to 

\begin{itemize}
\item design a passive lowpass filter circuit to filter noisy input signals and 
  
\item analyze the frequency response of a first-order passive circuit.  

\end{itemize}

\section{Parts}
\label{sec:parts}

\begin{enumerate}  
\item Breadboard,
\item Two $1~[\kilo\ohm]$ resistors,  
\item One $0.1~[\micro\farad]$ capacitor, and
\item One $50~[\milli\henry]$ inductor.
  
\end{enumerate}

\section{Background}
\label{sec:background}
Filtering is a technique to remove unwanted signals from a desired signal. A
filter is a device that implements a filtering technique. Figure~\ref{fig:fig6p33}
shows that a time-varying noisy (unwanted) signal, $v_i(t),$ is applied at the
input of a filter. A clean signal is generated at the filter output. %
%
\begin{figure}
  \begin{center}
    \begin{tikzpicture}
            \node[inner sep=0pt](inPlot){\includegraphics[width=0.48\textwidth,height=0.2\textheight]{matlabCodeHambley2018/OUT/fig6p33-Noisy}};
    \draw[->,ultra thick,blue]
    ($(inPlot.south)+(-1.5cm,0)$)--($(inPlot.south)+(-1.5cm,-0.5cm)$);      
    \end{tikzpicture}
  \end{center}

  % \begin{mdframed}[backgroundcolor=yellow!10,roundcorner=7pt,outerlinecolor=blue,outerlinewidth=1pt]
      \begin{center}
        \begin{circuitikz}[scale=0.8,american voltages]
          \tikzstyle{every node} = [font=\tiny]
          \draw
          (3*\smgrid,0) -- (2*\smgrid,0) to[short,o-*](0,0)node[ground]{} to[sV,v<=$v_i(t)$] (0,3*\smgrid) to[short,-o](2*\smgrid,3*\smgrid) --(3*\smgrid,3*\smgrid);
          \draw[fill=gray!20]
          (3*\smgrid,-0.5*\smgrid) rectangle (6*\smgrid,3.5*\smgrid) node[pos=0.5]{Filter};
          \draw
          (6*\smgrid,3*\smgrid) to[short,-o](7.5*\smgrid,3*\smgrid);
          \draw
          (6*\smgrid,0) to[short,-o](7.5*\smgrid,0) to[open,v<=$v_o(t)$](7.5*\smgrid,3*\smgrid);
        \end{circuitikz}
      \end{center}        
    % \end{mdframed}
    
  \begin{center}
    \begin{tikzpicture}
      \node[inner sep=0pt](outPlot){\includegraphics[width=0.48\textwidth,height=0.2\textheight]{matlabCodeHambley2018/OUT/fig6p33-Clean}};
    \draw[->,ultra thick,blue]
    ($(outPlot.north)+(1.5cm,+0.6cm)$) -- ($(outPlot.north) + (1.5cm,0)$);            
    \end{tikzpicture}
  \end{center}  
  \caption{Filtering noisy signal.}
  \label{fig:fig6p33}
\end{figure}
%
Filters can mainly be classified as analog and digital. Analog filters are
futher classified as passive and active. Passive filters are constructed using
passive electrical circuit elements, such as resistors, inductors, and
capacitors. Active filters include operational amplifiers (op-amps). Therefore,
passive filters may be used for high frequency applications since op-amps have
limited bandwidth. A certain range of frequency components, called
\emph{passband}, of the input signal can be passed through a filter circuit.
Depending on the frequency band of the input signal to be passed through the
filter relatively unattenuated, a filter circuit can be arranged in a variety of
ways to mainly yield lowpass, highpass, bandpass, and band-reject filters. Ideal
characteristics of these filters are shown in Figure~\ref{fig:fig6p32}. %
%
  \begin{figure}
    \centering
    \subfigure[][]{
      \label{fig:lowpassH}
    \begin{tikzpicture}%[scale=0.8]
      % \tikzstyle{every node} = [font=\tiny];
      \draw[step=0.5*\smgrid,gray,very thin](0,0) grid (2.5*\smgrid,2.5*\smgrid);
        \draw[thick,->](0,0) -- (0,2.5*\smgrid)node[anchor=south]{$|H(f)|$};
        \draw[thick,->](0,0) -- (2.5*\smgrid,0)node[anchor=west]{$f$};
        \draw[blue,very thick]
        (0,2*\smgrid)node[anchor=east]{$1$} -- (2*\smgrid,2*\smgrid) --(2*\smgrid,0) node[anchor=north]{$f_H$};
      \end{tikzpicture}
    }
    \subfigure[][]{
      \label{fig:highpassH}
    \begin{tikzpicture}%[scale=0.8]
      % \tikzstyle{every node} = [font=\tiny];
      \draw[step=0.5*\smgrid,gray,very thin](0,0) grid (2.5*\smgrid,2.5*\smgrid);
        \draw[thick,->](0,0) -- (0,2.5*\smgrid)node[anchor=south]{$|H(f)|$};
        \draw[thick,->](0,0) -- (2.5*\smgrid,0)node[anchor=west]{$f$};
        \draw[blue,very thick]
        (0,2*\smgrid)node[anchor=east]{$1$};
        \draw[blue,very thick]
        (0,0) -- (\smgrid,0) node[anchor=north]{$f_L$} --(\smgrid,2*\smgrid)--(2*\smgrid,2*\smgrid);        
      \end{tikzpicture}      
    }
    \subfigure[][]{
      \label{fig:bandpassH}
    \begin{tikzpicture}%[scale=0.8]
      % \tikzstyle{every node} = [font=\tiny];
      \draw[step=0.5*\smgrid,gray,very thin](0,0) grid (2.5*\smgrid,2.5*\smgrid);
        \draw[thick,->](0,0) -- (0,2.5*\smgrid)node[anchor=south]{$|H(f)|$};
        \draw[thick,->](0,0) -- (2.5*\smgrid,0)node[anchor=west]{$f$};
        \draw[blue,very thick]
        (0,2*\smgrid)node[anchor=east]{$1$};
        \draw[blue,very thick]
        (0,0) -- (\smgrid,0) node[anchor=north]{$f_L$} |-(2*\smgrid,2*\smgrid)--(2*\smgrid,0)node[anchor=north]{$f_H$}--(2.5*\smgrid,0);        
      \end{tikzpicture}      
    }
    \subfigure[][]{
      \label{fig:notchH}
    \begin{tikzpicture}%[scale=0.8]
      % \tikzstyle{every node} = [font=\tiny];
        \draw[step=0.5*\smgrid,gray,very thin](0,0) grid (2.5*\smgrid,2.5*\smgrid);
        \draw[thick,->](0,0) -- (0,2.5*\smgrid)node[anchor=south]{$|H(f)|$};
        \draw[thick,->](0,0) -- (2.5*\smgrid,0)node[anchor=west]{$f$};
        \draw[blue,very thick]
        (0,2*\smgrid)node[anchor=east]{$1$};
        \draw[blue,very thick]
        (0,2*\smgrid) -| (\smgrid,0) node[anchor=north]{$f_L$} --(2*\smgrid,0) node[anchor=north]{$f_H$} |-(2.5*\smgrid,2*\smgrid);        
      \end{tikzpicture}      
    }            
    % \includegraphics[width=0.6\textwidth]{figs/img/Hambley2018/fig6p4}
    \caption{Ideal filters~\subref{fig:lowpassH} lowpass,~\subref{fig:highpassH} highpass,~\subref{fig:bandpassH} bandpass, and~\subref{fig:notchH} band-reject.}
    \label{fig:fig6p32}
  \end{figure}
%  
  The frequency response is described using the transfer function $H(f)$ and the
  cutoff frequencies are denoted with $f_H$ and  $f_L.$ The number of independent capacitors and inductors determines the order (first-order, second-order, and so on) of a filter circuit. 
  
  \section{First-order Lowpass Filters}
\label{sec:first-order-lowpass}
Consider first-order circuits that contain an independent ac voltage source, a resistive element, and an energy storage element (a capacitor or an inductor). In particular, we shall build circuits that implement  first-order passive lowpass filters. Lowpass filters tend to pass low frequency input signal  components to the output and reject high frequency input signal components. Filters are used in a wide range of applications. For example, the telephone land lines employ lowpass filters.  A first-order passive lowpass filter can be built using %
%
\begin{enumerate}[a)]
\item an RC circuit and
  
\item an RL circuit. 
\end{enumerate}
%
We shall focus on testing first-order passive lowpass filters using both RC and RL circuits. Figure~\ref{fig:figure1-RC-Circuit} shows a series RC circuit that consists of an input (AC) voltage source connected in series with a resistor (R) and a capacitor (C).  %
%
\begin{figure}
  \centering
  \subfigure[][]{
\label{fig:figure1-RC-Circuit}    
\begin{circuitikz}[scale=1.2,american voltages]
  \draw
  (0,0) to[sV,l^=$v_i(t)$,fill=green!50] (0,4*\smgrid) to[R,v_>=$v_R(t)$, l^=$R$] (4*\smgrid,4*\smgrid);
  % draw capacitor
  \draw 
  (4*\smgrid,0) to[pC, v_<=$C$] (4*\smgrid,4*\smgrid);
  % draw ground
  \draw (0,0) to[short,-*](4*\smgrid,0) node[ground]{};
  %  Output
  \draw
  (4*\smgrid,4*\smgrid) to[short,-o](8*\smgrid,4*\smgrid);
  \draw 
  (4*\smgrid,0) to[short,-o](8*\smgrid,0);

  \draw % output voltage
  (8*\smgrid,0) to[open,v<=$v_C(t){=}v_o(t)$] (8*\smgrid,4*\smgrid); 
\end{circuitikz}
}
\subfigure[][]{
  \label{fig:figure1-RL-Circuit}
  \begin{circuitikz}[scale=1.2,american voltages]
    \draw  % source and inductor
    (0,0) to[sV,l^=$v_i(t)$,fill=green!50] (0,4*\smgrid) to[cute inductor,v_>=$v_L(t)$, l^=$L$] (4*\smgrid,4*\smgrid);
    % draw resistor
    \draw 
    (4*\smgrid,0) to[R, v_<=$R$] (4*\smgrid,4*\smgrid);
    % draw ground
    \draw (0,0) to[short,-*](4*\smgrid,0) node[ground]{};
    % Output
    \draw
    (4*\smgrid,4*\smgrid) to[short,-o](8*\smgrid,4*\smgrid);
    \draw 
    (4*\smgrid,0) to[short,-o](8*\smgrid,0);

    \draw % output voltage
    (8*\smgrid,0) to[open,v<=$v_R(t){=}v_o(t)$] (8*\smgrid,4*\smgrid); 
  \end{circuitikz}    
  }
  \caption{Passive lowpass filters:~\subref{fig:figure1-RC-Circuit} Series RC circuit and~\subref{fig:figure1-RL-Circuit} a series RL circuit.}
  \label{fig:passiveLPF}
\end{figure}
%
The circuit has a time-varying input voltage given by a general sinusoid $v_i(t) =A\cos(\omega t) + B\sin(\omega t),$ where $v_i(t)$ represents the input voltage at time $t\ge 0,$ $A$ and $B$ represent its amplitudes (peak voltages~[\si{\volt}]) of  cosine and sine components, respectively, and $\omega$ is the angular frequency in~[\si{\radian\per\second}]. Note that the angular frequency $\omega = 2\pi f,$ where $f$ is the frequency of the signal in [\si{\hertz}]. The filter output $v_o(t)$ is taken across the  capacitor (resistor) of the RC (RL) circuit shown in Figure~\ref{fig:figure1-RC-Circuit} [Figure~\ref{fig:figure1-RL-Circuit}].  Each circuit shown in Figure~\ref{fig:passiveLPF} can be represented using the state--space form as: %
%
\begin{subequations}
  \label{eq:stateSpaceRC}
\begin{align}
        \dot x(t)&\equiv \frac{\mathrm{d}x(t)}{\mathrm{dt}} =ax(t) + bu(t),\\
        y(t)& = cx(t) + du(t),
      \end{align}  
\end{subequations}

      %
where $x(t)\equiv y(t) \equiv v_o(t),$  $u(t) \equiv v_i(t),$ and the model parameters:
%
\begin{align*}
  \begin{cases}
    a=-\frac{1}{RC},~~b=\frac{1}{RC},~~c=1,~~\mathrm{and}~~d=0, \qquad\text{for the series RC circuit shown in Figure~\ref{fig:figure1-RC-Circuit}, and}\\
    a=-\frac{R}{L},~~b=\frac{R}{L},~~c=1,~~\mathrm{and}~~d=0, \qquad\text{for the series RL circuit shown in Figure~\ref{fig:figure1-RL-Circuit}.}    
  \end{cases}
\end{align*}

 Suppose that the Laplace transformations of the input, $v_i(t),$ and the output, $v_o(t),$ are $V_i(s)$ and $V_o(s),$ respectively, where $s$ is the Laplace variable. The Laplace transfer function, $G(s),$ of the circuits shown in Figure~\ref{fig:passiveLPF} is given by: %
%
 \begin{align}
   G(s) = 
   \begin{cases}
     \frac{1}{RCs + 1},\qquad\text{for the series RC circuit  shown in Figure~\ref{fig:figure1-RC-Circuit}, and}\\
     \frac{\frac{R}{L}}{s + \frac{R}{L}}, \qquad\text{for the series RL circuit  shown in Figure~\ref{fig:figure1-RL-Circuit}.}
   \end{cases}
  \label{eq:LTF-RC}
\end{align}
%
Substituting $s=j2\pi f$ in Equation~\eqref{eq:LTF-RC} yields the frequency transfer function: %
%
%
\begin{align}
  H(f) = \frac{1}{1 + j\left(\frac{f}{f_B}\right)},\qquad\text{where}
  \label{eq:FTF-RC}
\end{align}
%
break frequency, $f_B,$ is defined as: 
\begin{align*}
  f_B = 
\begin{cases}
  \frac{1}{2\pi RC}, \qquad\text{for the series RC circuit  shown in Figure~\ref{fig:figure1-RC-Circuit}, and}\\
  \frac{R}{2\pi L}, \qquad\text{for the series RL circuit  shown in Figure~\ref{fig:figure1-RL-Circuit}.}
\end{cases}  
\end{align*}
%
The magnitude expression of the frequency transfer function is given by: %
%
  \begin{align}
    \label{eq:RC-Mag}
    \left | H(f) \right | = \frac{1}{\sqrt{1+\left(\frac{f}{f_B}\right)^2}}
  \end{align}
%
and the Bode magnitude and phase expressions can be derived as: 
%
\begin{subequations}
  \label{eq:BodeRC}
  \begin{align}
    \label{eq:BodeFirstOrder-Mag}
    \left | H(f) \right |_{\si{\decibel}} &= -10\log_{10}\left[1+\left(\frac{f}{f_B}\right)^2\right],\\
            \label{eq:BodeFirstOrder-Phase}
    \angle{H(f)} & = -\angle{\left[1+j\left(\frac{f}{f_B}\right)\right]}.
  \end{align}
\end{subequations}
%
For a generic sinusoidal input signal $v_i(t) = A\cos(\omega t) + B\sin(\omega t),$ the time-domain expression of the output voltage $v_o(t)$ is: %
%
\begin{align}
  v_o(t) = \left(\sqrt{A^2+B^2}\right)\left | H(f)\right | \cos\left(\omega t + \phi_i+\phi_G\right),
  \label{eq:LPF-Out}
\end{align}
%
where $\phi_i = -\tan^{-1}(B/A)$ and $\phi_G = \angle{H(f)}.$

\section{Second-order Lowpass Filter}
\label{sec:second-order-lowpass}
Figure~\ref{fig:SOLPF} shows a second-order lowpass filter, where an inductor, $L,$ is added in series with the resistor of the first-order lowpass filter. %
%
%
  \begin{figure}
    \centering
      \begin{circuitikz}[american]
        % \tikzstyle{every node} = [font=\tiny];
        \draw (0,0) to[sV,v<=$v_i(t)$,fill=green!50](0,4*\smgrid) to[R,l=$R$]
        (4*\smgrid,4*\smgrid) to[cute inductor,l=$L$,-*](8*\smgrid,4*\smgrid)
        to[short,-o](10*\smgrid,4*\smgrid); \draw (0,0)node[ground]{}
        to[short,*-](8*\smgrid,0) to[pC,l=$C$](8*\smgrid,4*\smgrid); \draw
        (8*\smgrid,0) to[short,-o](10*\smgrid,0); \draw (10*\smgrid,0*\smgrid)
        to[open,v<=$v_o(t)$](10*\smgrid,4*\smgrid);
      \end{circuitikz}   
    \caption{Second-order lowpass filter.}
    \label{fig:SOLPF}
  \end{figure}
%
The  circuit shown in Figure~\ref{fig:SOLPF} can be represented using the state--space form as: %
%
\begin{subequations}
  \label{eq:stateSpaceSOLPF}
\begin{align}
        \dot{\mathbf{x}}(t)& =\mathbf{A}\mathbf{x}(t) + \mathbf{b}u(t),\\
        y(t)& = \mathbf{c}x(t) + du(t),
      \end{align}  
\end{subequations}
%
where $\mathbf{x}(t)\equiv [q(t),~i(t)]^T,$ $y(t)\equiv v_o(t),$  $u(t) \equiv v_i(t),$ and the model parameters:
%
\begin{align*}
    \mathbf{A}=
    \begin{bmatrix}
      0 & 1\\
      -\frac{1}{LC} & -\frac{R}{L}
    \end{bmatrix},~~
    \mathbf{b}=
    \begin{bmatrix}
      0\\
      \frac{1}{L}
    \end{bmatrix},~~
    \mathbf{c}=
    \begin{bmatrix}
      \frac{1}{C} & 0
    \end{bmatrix}
,~~\mathrm{and}~~d=0. 
\end{align*}
%
The Laplace transfer function, $G(s),$ of the circuits shown in Figure~\ref{fig:SOLPF} is given by: %
%
 \begin{align}
   G(s) = \frac{1}{LCs^2+ RCs + 1}
  \label{eq:LTF-SOLPF}
\end{align}
%
Substituting $s=j2\pi f$ in Equation~\eqref{eq:LTF-SOLPF} yields the frequency transfer function: %
%
%
\begin{align}
  H(f) =  \frac{1}{\left[1-\left(\frac{f}{f_0}\right)^2\right]+j\frac{1}{Q_s}\left(\frac{f}{f_0}\right)},
  \label{eq:FTF-RC}
\end{align}
%
where $f_0$ and $Q_s$ are the resonant frequency and the quality factor, respectively. The quantities are given by %
%
\begin{align*}
  f_0 = \frac{1}{2\pi\sqrt{LC}}\qquad\text{and}\qquad 
      Q_s = \frac{2\pi f_0 L}{R} = \frac{1}{2\pi f_0CR}.
\end{align*}
%
Clearly, the magnitude and phase expressions of the frequeny transfer function are derived as:  %
%
\begin{align*}
  \left | H(f) \right | = \frac{1}{\sqrt{\left[1-\left(\frac{f}{f_0}\right)^2\right]^2+\left(\frac{1}{Q_s}\frac{f}{f_0}\right)^2}}\qquad\text{and}\qquad
  \angle{H(f)} = - \angle\left[\left[1-\left(\frac{f}{f_0}\right)^2\right]+j\frac{1}{Q_s}\left(\frac{f}{f_0}\right)\right],
\end{align*}
%
respectively. 



\section{Prelab}
\label{sec:prelab}

You are to analyze the passive lowpass filters  shown in Figure~\ref{fig:passiveLPF} based on the theory presented in the previous section. Here, you will need to simulate both RC and RL circuits for specific parameter values.  


\begin{prelab}[Series RC circuit]{prelab:RC-circuit}
Given the circuit shown in Figure~\ref{fig:figure1-RC-Circuit} with $R = 1~[\kilo\ohm],$ and $C = 0.1~[\micro\farad].$ 
      \begin{enumerate}[(a)]
      \item  Represent the circuit in state-space form given by 
        % 
        \begin{align*}
          \dot x(t) &= a x(t) + bu(t),\\
          y(t) &= cx(t) + du(t),
        \end{align*}
        % 
        \textit{i.e.}, find the values of parameters $a,$ $b,$ $c,$ and $d.$ 
      \item Find the expression for the transfer function, $G(s) = \frac{V_o(s)}{V_i(s)},$ in the complex frequency (Laplace) domain.
        
      \item Find the expression for the frequency transfer function $H(f)$ and the value of the half--power frequency, $f_B,$ in [\hertz].
      \item Plot the magnitude and phase of the frequency transfer function, $H(f),$ versus frequency (in [\hertz]) using MATLAB. [Note: You will need to have two plots;  one plot for the magnitude and a second plot for the phase.]        
      \item  In one plot, show the input and the output of the circuit if $v_i(t) = 5\cos(10\pi t) + 5\cos(1000\pi t) +5\cos(5000\pi t),$ for $t\in[0,100t_s]~\second,$ where $t_s=\frac{1}{2f_{\mathrm{max}}}$ with $f_{\mathrm{max}}$ being the maximum frequency  (in~[\hertz])  of the input signal.
        \end{enumerate}
\end{prelab}

\begin{prelab}[Series RL circuit]{prelab:RL-circuit}
Given the circuit shown in Figure~\ref{fig:figure1-RL-Circuit} with $R = 1~[\kilo\ohm],$ and $L = 50~[\milli\henry].$ 
      \begin{enumerate}[(a)]
      \item  Represent the circuit in state-space form given by 
        % 
        \begin{align*}
          \dot x(t) &= a x(t) + bu(t),\\
          y(t) &= cx(t) + du(t),
        \end{align*}
        % 
        \textit{i.e.}, find the values of parameters $a,$ $b,$ $c,$ and $d.$ 
      \item Find the transfer function, $G(s) = \frac{V_o(s)}{V_i(s)},$ in the complex frequency (Laplace) domain.
        
      \item Find the frequency transfer function $H(f)$ and the value of the half--power frequency, $f_B,$ in [\hertz].
        
      \item Plot the magnitude and phase of the frequency transfer function, $H(f),$ versus frequency (in [\hertz]) using MATLAB. [Note: You will need to have two plots; one plot for the magnitude and a second plot for the phase.]
        
      \item  In one plot, show the input and the output of the circuit if $v_i(t) = 5\cos(10\pi t) + 5\cos(1000\pi t) +5\cos(5000\pi t),$ for $t\in[0,100t_s]~\second,$ where $t_s=\frac{1}{2f_{\mathrm{max}}}$ with $f_{\mathrm{max}}$ being the maximum frequency (in~[\hertz])  of the input signal.
        \end{enumerate}
\end{prelab}

\begin{prelab}[Series RLC circuit]{prelab:RLC-circuit}
Given the circuit shown in Figure~\ref{fig:SOLPF} with $R = 707~[\ohm],$ $L = 50~[\milli\henry],$ and $C = 200~[\nano\farad].$ 
      \begin{enumerate}[(a)]
      \item  Represent the circuit in state-space form given by 
        %
        \begin{align*}
          \dot{\mathbf{x}}(t)& =\mathbf{A}\mathbf{x}(t) + \mathbf{b}u(t),\\
          y(t)& = \mathbf{c}x(t) + du(t),
        \end{align*}
        %
        \textit{i.e.}, find the values of parameters $\mathbf{A},$ $\mathbf{b},$ $\mathbf{c},$ and $d.$ 
      \item Find the transfer function, $G(s) = \frac{V_o(s)}{V_i(s)},$ in the complex frequency (Laplace) domain.
        
      \item Find the frequency transfer function $H(f)$ and the value of the resonant frequency, $f_0,$ in [\hertz] and the quality factor $Q_s.$
        
      \item Plot the magnitude and phase of the frequency transfer function, $H(f),$ versus frequency (in $\hertz$) using MATLAB. [Note: You will need to have two plots; one plot for the magnitude and a second plot for the phase.]
        
      \item  In one plot, show the input and the output of the circuit if $v_i(t) = 5\cos(200\pi t) + 5\cos(2000\pi t) +5\cos(20000\pi t),$ for $t\in[0,100t_s]~\second,$ where $t_s=\frac{1}{2f_{\mathrm{max}}}$ with $f_{\mathrm{max}}$ being the maximum frequency (in~[\hertz])  of the input signal.
        \end{enumerate}
 \end{prelab}
      

\section{Laboratory Work}

You will test the passive lowpass filters simulated in the prelab. 
\subsection{Part~1}
\label{sec:part1}


\begin{enumerate}

 
\item Measure the values of the $1~[\kilo\ohm]$ resistor $(R)$ and the $0.1~[\micro\farad]$ capacitor $(C).$ Then, complete the following table.

  \begin{center}
    \begin{tabular}{c|c|c}
      \toprule
      Quantity &  Ideal & Measured\\
      \toprule
      $R$ & $\ldots$ & $\ldots$\\   %|| R = 
      $C$ & $\ldots$ & $\ldots$\\   %|| C = 
      \bottomrule
    \end{tabular}    
  \end{center}
  
\item Construct the circuit shown in Figure~\ref{fig:figure1-RC-Circuit} using the components measured in the previous step. 

\item  Turn on the function generator and generate a $v_i(t) = 5\sin(2000\pi t)~[\volt]$ sinusoidal waveform and connect the function generator to the input terminals of the circuit that you built. 

  
\item Connect the oscilloscope (two channels) to the input and output terminals, and the record the data in the table below. %
%
    \begin{center}
    \begin{tabular}{c|c|c|c|c|c}
      \toprule
      Frequency &  $V_{\mathrm{PP,input}}$ & $V_{\mathrm{PP,output}}$ & Phase difference & $\left |H(f) \right |$ & $\left |H(f) \right |_{\decibel}$\\
      \toprule
      ~ & ~ & ~ & ~ & ~ & ~\\
      \bottomrule
    \end{tabular}    
  \end{center}
%  
  

  
\item Change the frequency of the input signal (waveform) and complete the following table
%
    \begin{center}
    \begin{tabular}{l|c|c|c|c|c}
      \toprule
      Frequency &  $V_{\mathrm{PP,input}}$ & $V_{\mathrm{PP,output}}$ & Phase difference & $\left |H(f) \right |$ & $\left |H(f) \right |_{\decibel}$\\
      \toprule
      $20~[\hertz]$ & ~ & ~ & ~ & ~ & ~\\
      $200~[\hertz]$ & ~ & ~ & ~ & ~ & ~\\
      $2~[\kilo\hertz]$ & ~ & ~ & ~ & ~ & ~\\
      $20~[\kilo\hertz]$ & ~ & ~ & ~ & ~ & ~\\
      $200~[\kilo\hertz]$ & ~ & ~ & ~ & ~ & ~\\
      $2~[\mega\hertz]$ & ~ & ~ & ~ & ~ & ~\\
      $20~[\mega\hertz]$ & ~ & ~ & ~ & ~ & ~\\      
      \bottomrule
    \end{tabular}    
  \end{center}
%  
\item Compare the measured values of the filter gain $\left|H(f)\right |_{\decibel}$ and the theoretical values of $\left|H(f)\right |_{\decibel}$ using Equation~\eqref{eq:BodeFirstOrder-Mag} and, then discuss any discrepancy.   

 \end{enumerate}


 \subsection{Part~2}
\label{sec:part2}


\begin{enumerate}

 
\item Measure the values of the $1~[\kilo\ohm]$ resistor $(R)$ and the $50~[\milli\henry]$ inductor $(L).$ Then, complete the following table.

  \begin{center}
    \begin{tabular}{c|c|c}
      \toprule
      Quantity &  Ideal & Measured\\
      \toprule
      $R$ & $\ldots$ & $\ldots$\\   %|| R = 
      $L$ & $\ldots$ & $\ldots$\\   %|| L = 
      \bottomrule
    \end{tabular}    
  \end{center}
  
\item Construct the circuit shown in Figure~\ref{fig:figure1-RL-Circuit} using the components measured in the previous step. 

\item  Turn on the function generator and generate a $v_i(t) = 5\sin(2000\pi t)~[\volt]$ sinusoidal  waveform and connect the function generator to the input terminals of the circuit that you built. 

  
\item Connect the oscilloscope (two channels) to the input and output terminals, and then record the data in the following table. %
%
    \begin{center}
    \begin{tabular}{c|c|c|c|c|c}
      \toprule
      Frequency &  $V_{\mathrm{PP,input}}$ & $V_{\mathrm{PP,output}}$ & Phase difference & $\left |H(f) \right |$ & $\left |H(f) \right |_{\decibel}$\\
      \toprule
      ~ & ~ & ~ & ~ & ~ & ~\\
      \bottomrule
    \end{tabular}    
  \end{center}
%  
    
\item Change the frequency of the input signal (waveform) and complete the following table. %
%
    \begin{center}
    \begin{tabular}{l|c|c|c|c|c}
      \toprule
      Frequency &  $V_{\mathrm{PP,input}}$ & $V_{\mathrm{PP,output}}$ & Phase difference & $\left |H(f) \right |$ & $\left |H(f) \right |_{\decibel}$\\
      \toprule
      $20~[\hertz]$ & ~ & ~ & ~ & ~ & ~\\
      $200~[\hertz]$ & ~ & ~ & ~ & ~ & ~\\
      $2~[\kilo\hertz]$ & ~ & ~ & ~ & ~ & ~\\
      $20~[\kilo\hertz]$ & ~ & ~ & ~ & ~ & ~\\
      $200~[\kilo\hertz]$ & ~ & ~ & ~ & ~ & ~\\
      $2~[\mega\hertz]$ & ~ & ~ & ~ & ~ & ~\\
      $20~[\mega\hertz]$ & ~ & ~ & ~ & ~ & ~\\      
      \bottomrule
    \end{tabular}    
  \end{center}
%  
\item Compare the measured values of the filter gain $\left|H(f)\right |_{\decibel}$ and the theoretical values of $\left|H(f)\right |_{\decibel}$ using Equation~\eqref{eq:BodeFirstOrder-Mag}, and then discuss any discrepancy.   

 \end{enumerate}
 


%%% Local Variables:
%%% mode: latex
%%% TeX-master: "../../labBookMechatronics-V2"
%%% End:
