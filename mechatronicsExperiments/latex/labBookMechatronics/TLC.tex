\section{Objectives}
By the end of this laboratory assignment, students are expected to learn how to 

\begin{itemize}

\item use C libraries for interfacing general-purpose input-output (GPIO) pins with a single-board computer and   
  
\item implement a simple traffic light control algorithm using onboard LEDs of a single-board computer.  
 
  
\end{itemize}

\section{Parts}
\label{sec:partsTLC}
The following parts are required to conduct this laboratory assignment. %
%
\begin{enumerate}
\item BBBlue board with the latest OS (debian) image installed
\item One micro-USB cable  
\end{enumerate}

\section{Background}
\label{sec:background}

This laboratory assignment is primarily meant to use a few C library functions
of the robotics cape package, \emph{librobotcontrol}, for controlling onboard
LEDs and GPIO pins on the BBBlue embedded computer (SBC). The \emph{librobotcontrol} package is pre-installed in the BBBlue board. In this laboratory assignment, we shall focus on implementing  algorithms that emulate the following applications:
%
\begin{enumerate}
    \item Flashing red light and 
    \item Traffic light controller at an intersection of two one-way roads.
\end{enumerate}
%
Before implementing the algorithms that emulate the above applications, it is important to review on how to use the light emitting diodes (LEDs) and general-purpose input-output (GPIO) pins of the BBBlue embedded computer. 

\subsection{Light-emitting Diode (LED)}
\label{sec:LEDs}

The BBBlue board has eleven light-emitting diodes (LEDs) that are accessed using the following application programming interfaces (APIs).
\begin{center}
\begin{verbatim}
RC_LED_GREEN
RC_LED_RED
RC_LED_USR0
RC_LED_USR1
RC_LED_USR2
RC_LED_USR3
RC_LED_BAT25
RC_LED_BAT50
RC_LED_BAT75
RC_LED_BAT100
RC_LED_WIFI
\end{verbatim}
\end{center}
%
Each LED on the BBBlue board is labeled so that it is easily identified with the API name mentioned above. These LEDs are controlled using the following functions (see the header file \emph{rc/led.h}): %
%
\begin{verbatim}
  rc_led_set (rc_led_t led, int value)
  rc_led_cleanup (void)
\end{verbatim}
%
The \emph{rc\_led\_set} function is used to turn an LED ON or OFF. To call this function, the name of the LED to control is passed as the \emph{led} argument, and the desired state is passed as the \emph{value} ($0$ for OFF, nonzero for ON). For example, the following statement would turn on the onboard green LED: %
%
\begin{verbatim}
  rc_led_set (RC_LED_GREEN, 1);
\end{verbatim}
%
To properly close out the LEDs before exiting the program, the \emph{rc\_led\_cleanup} function should be called at the end of the program. %

\subsection{GPIO Pins}
\label{sec:GPIOPins}
The BBBlue board has several general-purpose input-output (GPIO) pins that can
be used to receive inputs or drive outputs. The pins accessible on ports
\emph{GP0} and \emph{GP1} are shown in~\autoref{fig:GPIO_pins}. These pins can
be connected to external circuits with pushbuttons or LEDs, for instance, for providing additional functionality of the BBBlue embedded computer. %
%
\begin{figure}[htbp]
  \centering
  \includegraphics[width=0.5\textwidth]{figs/img/GPIO_pins.png}
  \caption{BBBlue GPIO Ports GP0 and GP1}
  \label{fig:GPIO_pins}
\end{figure}
%
The GPIO pins are controlled using the following functions (see the header file \emph{rc/gpio.h}): %
\begin{verbatim}
  rc_gpio_init (int chip, int pin, int handle_flags)
  rc_gpio_get_value (int chip, int pin)
  rc_gpio_set_value (int chip, int pin, int value)
  rc_gpio_cleanup (int chip, int pin)
\end{verbatim}
%
Each of these functions requires inputs of both a chip and a pin. In~\autoref{fig:GPIO_pins}, the GPIO pins are labeled as \emph{GPIOx\_y}, where \emph{x} is the chip number, and \emph{y} is the pin number. For example, pin $4$ of \emph{GP1} is on pin $1$ of chip $3.$ When the pin is initialized, it must be configured as either an input or an output by passing either \emph{GPIOHANDLE\_REQUEST\_INPUT} or \emph{GPIOHANDLE\_REQUEST\_OUTPUT} for the \emph{handle\_flags} parameter of \emph{rc\_gpio\_init}. For example, the following statement would be used to set pin $4$ of GP1 as an output: %
%
\begin{verbatim}
  rc_gpio_init (3, 1, GPIOHANDLE_REQUEST_OUTPUT);
\end{verbatim}
%
The value of a pin configured as an input can be obtained as the return value of the \emph{rc\_gpio\_get\_value} function. To set the output value of a pin configured as an output, the \emph{rc\_gpio\_set\_value} function is used. If a zero is passed as the value, the GPIO pin is set to OFF (inactive), while a nonzero value turns the pin ON (active). To properly close out the GPIO pin before exiting, the \emph{rc\_gpio\_cleanup} function is to be called at the end of the program on each GPIO pin that was used.

Note that LED\_RED and LED\_GRN pins of GP1 are connected to the onboard red and green LEDs, therefore, are controlled using the LED functions rather than the GPIO functions. See~\autoref{sec:LEDs} for details on using the LEDs.

\subsection{Flashing Red Light}
\label{sec:FlashingRedLight}
A flashing red light turns ON and OFF at a certain frequency. It is normally used at an  intersection of roads where traffic comes to a complete stop. A circuit that emulates a flashing red light is shown in~\autoref{fig:flashingLED1}.  A square wave voltage $v_s(t)$ for $t\ge 0$ mimics ON-OFF signals provided by the BBBlue embedded computer. 
%
\begin{figure}
    \centering
    \begin{circuitikz}[american]
      \draw (0,0) to[sqV,v<=$v_s(t)$]
      (0,4*\smgrid) to[R,v>=$R$](5*\smgrid,4*\smgrid) to[full led,o-o,l^=~~~LED
      (Red),v>=~](5*\smgrid,0) to[short,-*] (0,0) node[ground]{};
    \end{circuitikz}
    \caption{Circuit for emulating a flashing light.}
    \label{fig:flashingLED1}
\end{figure}
%
Therefore, the input is a binary voltage that is to be sent to turn ON and OFF
of an LED using an algorithm to be implemented on the BBBlue embedded computer.
The LED can be chosen from one of the onboard LEDs or the ON OFF signal can be
sent out an external circuit (an LED connected with a $330~[\ohm]$ resistor
connected in series on a breadboard, for example) through a GPIO output pin.
Note that resistor is not needed to be connected if an onboard LED is used. If an
external LED is used, a GPIO pin is to be connected with an LED through a
resistor connected in series much like the one shown in
Figure~\ref{fig:flashingLED1}.

\subsection{Traffic Light Controller}
\label{sec:voltageDivider}
In this part, a simple traffic light control logic is to be implemented on the BBBlue embedded computer. For simplicity, we will be considering traffic lights at the intersection
of two one-way roads: East-West (EW) and North-South (NS) as shown in
\autoref{fig:intersection1}. %
%
\begin{figure}
  \centering
  \begin{tikzpicture}[scale=0.6]
    % East-West
    \draw[very thick]
    (-5,1) -- (5,1)node[anchor=west]{\textcolor{ForestGreen}{$\bullet$~\textbf{Green}}};
    \draw
    (5,0) node[anchor=west]{\textcolor{YellowOrange}{$\bullet$~\textbf{Yellow}}};
    \draw[very thick,red,dashed]
    (-5,-1) -- (5,-1)node[anchor=west]{\textcolor{BrickRed}{$\bullet$~\textbf{Red}}};
    \draw[ultra thick,-stealth]
    (-5,0)node[anchor=east]{East} --(-3,0);
    \draw
    (8,0) node[anchor=west]{West};

    % North-South
    \draw[very thick]
    (-1,5) -- (-1,-5)node[anchor=south,rotate=90]{\textcolor{ForestGreen}{$\bullet$~\textbf{Green}}};
    \draw
    (0.5,-5) node[anchor=south,rotate=90]{\textcolor{YellowOrange}{$\bullet$~\textbf{Yellow}}};    
    \draw[very thick,red,dashed]
    (1,5) -- (1,-5)node[anchor=north,rotate=90]{\textcolor{BrickRed}{$\bullet$~\textbf{Red}}};
    \draw[ultra thick,-stealth]
    (0,5)node[anchor=south]{North} --(0,3);
    \draw
    (0,-8) node[anchor=north]{South};
  \end{tikzpicture}
  \caption{Traffic lights at an intersection of two one-way roads.}
  \label{fig:intersection1}
\end{figure}
%
Three LEDs are used to indicate the traffic signals
for each road: one Green, one Yellow, and one Red. At any given time, there is
an \emph{active direction} with its lights set to green and an \emph{inactive
  direction} with its lights set to red. In the active direction, the duration
of Green, Yellow, and Red lights are $10~[\second]$ (state\#1), $4~[\second]$
(state\#2) and $2~[\second]$ (state\#3), respectively. The duration of red light
in the inactive direction is $16~[\second].$ The active and inactive directions
are then switched. A full light cycle passes through three states as described
in \autoref{tab:traffic1}, where logic $1(0)$ represent that the corresponding
LED is ON (OFF). %
\begin{table}%[htbp]
\caption{Traffic light cycle.}
\label{tab:traffic1}
\centering
  \begin{tabular}{c|c|c|c}
  \toprule
  Direction & State~\#1 ($10~[\second]$) & State~\#2 ($4~[\second]$) & State~\#3 ($2~[\second]$)\\
  \toprule
  Active (\textcolor{ForestGreen}{Green}, \textcolor{YellowOrange}{Yellow}, \textcolor{BrickRed}{Red}) & (1, 0, 0) & (0, 1, 0) & (0, 0, 1)\\
  \hline
  Inactive (\textcolor{ForestGreen}{Green}, \textcolor{YellowOrange}{Yellow},\textcolor{BrickRed}{Red}) & (0, 0, 1) & (0, 0, 1) & (0, 0, 1)\\
  \bottomrule
  \end{tabular}
\end{table}


\begin{prelab}[Flashing RED light]{prelab:flashingRedLight}
  Suppose that the LED, RC\_LED\_RED, of the BBBlue board is used  to mimic a flashing red light at an intersection of roads. Write a C program for a BBBlue board so that it practically flashes the LED at the frequency of $0.5~[\hertz].$ The program is to perform the  following operations. %
  %
  \begin{enumerate}
  \item When PAU button is pressed, the LED practically flashes at the frequency of $0.5~[\hertz].$
  \item When MOD button is pressed, the LED stops flashing (LED is OFF) and the program terminates.      
  \end{enumerate}
  %
\end{prelab}




\section{Laboratory Work}

Write a C program for a BBBlue board so that it practically implements a traffic
light control logic  at an intersection of two one-way roads as illustrated in Figure~\ref{fig:intersection1}. Use the
onboard LEDs of the BBBlue to configure the traffic lights as given in
\autoref{tab:tlc2}. %
%
\begin{table}
  \centering
  \caption{LED configurations for implementing traffic light controller. }
  \label{tab:tlc2}  
  \begin{tabular}{l|l}
    \toprule   
    East--West & North--South\\
    \toprule
    RC\_LED\_GREEN $\to$ Green & RC\_LED\_USR1 $\to$ Green\\   	    
    RC\_LED\_USR0 $\to$ Yellow  	  & RC\_LED\_USR2 $\to$ Yellow\\  
    RC\_LED\_RED 	$\to$ Red          & RC\_LED\_USR3 $\to$ Red\\  
    \bottomrule    
  \end{tabular}
\end{table}
%
State durations of traffic lights are given in \autoref{tab:traffic1}. The program is to perform the following operations.


\begin{enumerate}
\item Once the PAU button is pressed, the LEDs in active and inactive directions
  will start to light as per the state durations given in \autoref{tab:traffic1}.
\item When the MOD button is pressed, all LEDs turn OFF and the program terminates.
\end{enumerate}

 

%%% Local Variables:
%%% mode: latex
%%% TeX-master: "../../labBookMechatronics-V2"
%%% End:
