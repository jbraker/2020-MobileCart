\section{Objectives}
By the end of this laboratory assignment, students are expected to learn how to 

\begin{itemize}

\item use C libraries for interfacing general-purpose input-output (GPIO) pins with a single-board computer and   
  
\item implement a simple traffic light control algorithm using onboard LEDs of a single-board computer.  
 
  
\end{itemize}

\section{Parts}
\label{sec:partsTLC}
The following parts are required to conduct this laboratory assignment. %
%
\begin{enumerate}
\item BBBlue board with the latest OS (debian) image installed
\item One micro-USB cable  
\end{enumerate}

\section{Background}
\label{sec:background}

This laboratory assignment is primarily meant to use some C library functions of the robotics cape package, \emph{librobotcontrol}, for controlling onboard LEDs on the BBBlue single-board computer (SBC).  The \emph{librobotcontrol} package is pre-installed in the BBBlue board. The BBBlue board has eleven LEDs that are accessed using the following application programming interfaces (APIs).
\begin{center}
\begin{verbatim}
RC_LED_GREEN 	
RC_LED_RED 	
RC_LED_USR0 	
RC_LED_USR1 	
RC_LED_USR2 	
RC_LED_USR3 	
RC_LED_BAT25 	
RC_LED_BAT50 	
RC_LED_BAT75 	
RC_LED_BAT100 	
RC_LED_WIFI 
\end{verbatim}  
\end{center}
%
Each LED on the BBBlue board is labeled so that it is easily identfied with the API name mentioned above. In this laboratory assignment, we shall focus on implementing  algorithms that emulate the following applications:
%
\begin{enumerate}
    \item Flashing red light and 
    \item Traffic light controller at an intersection of two one-way roads.
\end{enumerate}
%

Before implementing the algorithms that emulate the above applications, it is important to review the working principles of these applications. 

\subsection{Flashing Red Light}
\label{sec:FlashingRedLight}
A flashing red light turns on and off at a certain frequency. It is normally used at an  intersection of roads where traffic comes to a complete stop. A circuit that emulates a flashing red light is shown in Figure~\ref{fig:flashingLED1}.  A square wave voltage $v_s(t)$ for $t\ge 0$ mimics ON-OFF signals provided by the SBC. 
%
\begin{figure}
    \centering
    \begin{circuitikz}[scale=1.2,american voltages]
      \draw (0,0) to[sqV,v<=$v_s(t)$ (binary input from SBC),fill=green!50]
      (0,4*\smgrid) to[R,v>=$R$](5*\smgrid,4*\smgrid) to[full led,o-o,l^=~~~LED
      (Red),v>=~](5*\smgrid,0) to[short,-*] (0,0) node[ground]{};
    \end{circuitikz}
    \caption{Circuit for emulating a flashing light.}
    \label{fig:flashingLED1}
\end{figure}
%
Therefore, the input is a binary voltage that is to be sent to turn on and off of an LED using an algorithm to be implemented on BBBlue. The LED can be chosen from one of the onboard LEDs. Note that resistor is not needed to be connected if an onbard LED is used. 



\subsection{Traffic Light Controller}
\label{sec:voltageDivider}
In this part, a simple traffic light control logic is to be implemented on a SBC. For simplicity, we will be considering traffic lights at the intersection of two one-way roads: East-West (EW) and North-South (NS) as shown in Figure~\ref{fig:intersection1}. Three LEDs are used to indicate the traffic signals for each road: one Green, one Yellow, and  one Red. At any given time, there is an \emph{active direction} with its lights set to green  and an \emph{inactive direction} with its lights set to red. In the active direction, the duration of Green, Yellow, and Red lights are $10~[\second]$ (state\#1), $4~[\second]$ (state\#2) and $2~[\second]$ (state\#3), respectively.  The duration of Red light in the inactive direction is $16~[\second].$ The active and inactive directions are then switched. A full light cycle passes through three states as described in Table~\ref{tab:traffic1}, where logic $1(0)$ represent that the corresponding LED  is ON (OFF). %
%
\begin{figure}
  \centering
  \begin{tikzpicture}[scale=0.6]
    % East-West
    \draw[very thick]
    (-5,1) -- (5,1)node[anchor=west]{\textcolor{ForestGreen}{$\bullet$~\textbf{Green}}};
    \draw
    (5,0) node[anchor=west]{\textcolor{YellowOrange}{$\bullet$~\textbf{Yellow}}};
    \draw[very thick,red,dashed]
    (-5,-1) -- (5,-1)node[anchor=west]{\textcolor{BrickRed}{$\bullet$~\textbf{Red}}};
    \draw[ultra thick,->]
    (-5,0)node[anchor=east]{East} --(-3,0);
    \draw
    (8,0) node[anchor=west]{West};

    % North-South
    \draw[very thick]
    (-1,5) -- (-1,-5)node[anchor=south,rotate=90]{\textcolor{ForestGreen}{$\bullet$~\textbf{Green}}};
    \draw
    (0.5,-5) node[anchor=south,rotate=90]{\textcolor{YellowOrange}{$\bullet$~\textbf{Yellow}}};    
    \draw[very thick,red,dashed]
    (1,5) -- (1,-5)node[anchor=north,rotate=90]{\textcolor{BrickRed}{$\bullet$~\textbf{Red}}};
    \draw[ultra thick,->]
    (0,5)node[anchor=south]{North} --(0,3);
    \draw
    (0,-8) node[anchor=north]{South};
  \end{tikzpicture}
  \caption{Traffic lights at an intersection of two one-way roads.}
  \label{fig:intersection1}
\end{figure}

\begin{table}%[htbp]
\caption{Traffic light cycle.}
\label{tab:traffic1}
\centering
\begin{tabular}{c|c|c|c}
\toprule
Direction & State~\#1 ($10~[\second]$) & State~\#2 ($4~[\second]$) & State~\#3 ($2~[\second]$)\\
\toprule
Active (\textcolor{ForestGreen}{Green}, \textcolor{YellowOrange}{Yellow}, \textcolor{BrickRed}{Red}) & (1, 0, 0) & (0, 1, 0) & (0,0,1)\\
\hline
Inactive (\textcolor{ForestGreen}{Green}, \textcolor{YellowOrange}{Yellow},\textcolor{BrickRed}{Red}) & (0, 0, 1) & (0, 0, 1) & (0, 0, 1)\\
\bottomrule
\end{tabular}

\end{table}


\begin{prelab}[Flashing RED light]{prelab:flashingRedLight}
  Suppose that the LED, RC\_LED\_RED, of the BBBlue board is used  to mimic a flashing Red light at an intersection of roads. You are to write a C program in a BBBlue board so that it practically flashes the LED at the frequency of $0.5~[\hertz].$ The program should perform the  following operations.
  \begin{enumerate}
  \item When PAU button is pressed, the LED practically flashes at the frequency of $0.5~[\hertz].$
  \item When MOD button is pressed, the LED stops flashing (LED is OFF) and the program terminates.      
  \end{enumerate}
\end{prelab}




\section{Laboratory Work}

You are to write a C program in a BBBlue board so that it practically implements a traffic light control logic at an intersection of  two one-way roads. Use the onboard LEDs of the BBBlue to configure the traffic lights as illustrated in Table~\ref{tab:tlc2}. %
%
\begin{table}
  \centering
  \caption{LED configurations for implementing traffic light controller. }
  \label{tab:tlc2}  
  \begin{tabular}{l|l}
    \toprule   
    East--West & North--South\\
    \toprule
    RC\_LED\_GREEN $\to$ Green & RC\_LED\_USR1 $\to$ Green\\   	    
    RC\_LED\_USR0 $\to$ Yellow  	  & RC\_LED\_USR2 $\to$ Yellow\\  
    RC\_LED\_RED 	$\to$ Red          & RC\_LED\_USR3 $\to$ Red\\  
    \bottomrule    
  \end{tabular}
\end{table}
%
State durations of traffic lights are given in  Table~\ref{tab:traffic1}. The program should perform the  following operations.


\begin{enumerate}
\item When PAU button is pressed, the LEDs in active and inactive directions
  will flash as per the state durations given in Table~\ref{tab:traffic1}.
\item When MOD button is pressed, all LEDs are OFF and the program terminates.
\end{enumerate}
 

%%% Local Variables:
%%% mode: latex
%%% TeX-master: "../../labBookMechatronics-V2"
%%% End:
