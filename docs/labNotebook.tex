%%%%%%%%%%%%%%%%%%%%%%%%%%%%%%%%%%%%%%%%
% Compact Laboratory Book
% LaTeX Template
% Version 1.0 (4/6/12)
%
% This template has been downloaded from:
% http://www.LaTeXTemplates.com
%
% Original author:
% Joan Queralt Gil (http://phobos.xtec.cat/jqueralt) using the labbook class by
% Frank Kuster (http://www.ctan.org/tex-archive/macros/latex/contrib/labbook/)
%
% License:
% CC BY-NC-SA 3.0 (http://creativecommons.org/licenses/by-nc-sa/3.0/)
%
% Important note:
% This template requires the labbook.cls file to be in the same directory as the
% .tex file. The labbook.cls file provides the necessary structure to create the
% lab book.
%
% The \lipsum[#] commands throughout this template generate dummy text
% to fill the template out. These commands should all be removed when 
% writing lab book content.
%
% HOW TO USE THIS TEMPLATE 
% Each day in the lab consists of three main things:
%
% 1. LABDAY: The first thing to put is the \labday{} command with a date in 
% curly brackets, this will make a new section showing that you are working
% on a new day.
%
% 2. EXPERIMENT/SUBEXPERIMENT: Next you need to specify what 
% experiment(s) and subexperiment(s) you are working on with a 
% \experiment{} and \subexperiment{} commands with the experiment 
% shorthand in the curly brackets. The experiment shorthand is defined in the 
% 'DEFINITION OF EXPERIMENTS' section below, this means you can 
% say \experiment{pcr} and the actual text written to the PDF will be what 
% you set the 'pcr' experiment to be. If the experiment is a one off, you can 
% just write it in the bracket without creating a shorthand. Note: if you don't 
% want to have an experiment, just leave this out and it won't be printed.
%
% 3. CONTENT: Following the experiment is the content, i.e. what progress 
% you made on the experiment that day.
%
%%%%%%%%%%%%%%%%%%%%%%%%%%%%%%%%%%%%%%%%%

%----------------------------------------------------------------------------------------
%    PACKAGES AND OTHER DOCUMENT CONFIGURATIONS
%----------------------------------------------------------------------------------------                               

% \UseRawInputEncoding

\documentclass[fontsize=11pt, % Document font size
                             paper=letter, % Document paper type
                             %twoside, % Shifts odd pages to the left for easier reading when printed, can be changed to oneside
                             openany, % chapters can start on any page
                             captions=tableheading,
                             index=totoc,
                             hyperref]{labbook}

%\documentclass[idxtotoc,hyperref,openany]{labbook} % 'openany' here removes the
  
\usepackage[bottom=10em]{geometry} % Reduces the whitespace at the bottom of the page so more text can fit

\usepackage[english]{babel} % English language
\usepackage{lipsum} % Used for inserting dummy 'Lorem ipsum' text into the template

\usepackage[utf8]{inputenc} % Uses the utf8 input encoding
\usepackage[T1]{fontenc} % Use 8-bit encoding that has 256 glyphs

\usepackage[osf]{mathpazo} % Palatino as the main font
\linespread{1.05}\selectfont % Palatino needs some extra spacing, here 5% extra
\usepackage[scaled=.88]{beramono} % Bera-Monospace
\usepackage[scaled=.86]{berasans} % Bera Sans-Serif

\usepackage{booktabs,array} % Packages for tables

\usepackage{amsmath} % For typesetting math
\usepackage{graphicx} % Required for including images
\usepackage{etoolbox}
\usepackage[norule]{footmisc} % Removes the horizontal rule from footnotes
\usepackage{lastpage} % Counts the number of pages of the document
\usepackage{float}

\usepackage[ruled, vlined, linesnumbered]{algorithm2e} % For algorithms


\usepackage[dvipsnames]{xcolor}  % Allows the definition of hex colors
\usepackage{epstopdf}
\epstopdfsetup{suffix={}}
\definecolor{titleblue}{rgb}{0.16,0.24,0.64} % Custom color for the title on the title page
\definecolor{linkcolor}{rgb}{0,0,0.42} % Custom color for links - dark blue at the moment

\addtokomafont{title}{\Huge\color{titleblue}} % Titles in custom blue color
\addtokomafont{chapter}{\color{OliveGreen}} % Lab dates in olive green
\addtokomafont{section}{\color{Sepia}} % Sections in sepia
\addtokomafont{pagehead}{\normalfont\sffamily\color{gray}} % Header text in gray and sans serif
\addtokomafont{caption}{\footnotesize\itshape} % Small italic font size for captions
\addtokomafont{captionlabel}{\upshape\bfseries} % Bold for caption labels
\addtokomafont{descriptionlabel}{\rmfamily}
\setcapwidth[c]{10cm} % Center align caption text
\setkomafont{footnote}{\sffamily} % Footnotes in sans serif

\deffootnote[4cm]{4cm}{1em}{\textsuperscript{\thefootnotemark}} % Indent footnotes to line up with text

\DeclareFixedFont{\textcap}{T1}{phv}{bx}{n}{1.5cm} % Font for main title: Helvetica 1.5 cm
\DeclareFixedFont{\textaut}{T1}{phv}{bx}{n}{0.8cm} % Font for author name: Helvetica 0.8 cm

\usepackage{scrhack}
\usepackage[headsepline]{scrlayer-scrpage} % Provides headers and footers configuration
\pagestyle{scrheadings} % Print the headers and footers on all pages
\clearscrheadfoot % Clean old definitions if they exist

\automark[chapter]{chapter}
\ohead{\headmark} % Prints outer header

\setlength{\headheight}{25pt} % Makes the header take up a bit of extra space for aesthetics
\addtokomafont{headsepline}{\color{lightgray}} % Colors the rule under the header light gray

\ofoot[\normalfont\normalcolor{\thepage\ |\  \pageref{LastPage}}]{\normalfont\normalcolor{\thepage\ |\  \pageref{LastPage}}} % Creates an outer footer of: "current page | total pages"

% These lines make it so each new lab day directly follows the previous one i.e. does not start on a new page - comment them out to separate lab days on new pages
\makeatletter
\patchcmd{\addchap}{\if@openright\cleardoublepage\else\clearpage\fi}{\par}{}{}
\makeatother
\renewcommand*{\chapterpagestyle}{scrheadings}

% These lines make it so every figure and equation in the document is numbered consecutively rather than restarting at 1 for each lab day - comment them out to remove this behavior
\usepackage{chngcntr}
\counterwithout{figure}{labday}
\counterwithout{equation}{labday}

% Hyperlink configuration
\usepackage[
    pdfauthor={Jason Braker}, % Your name for the author field in the PDF
    pdftitle={Laboratory Journal}, % PDF title
    pdfsubject={mobileCart}, % PDF subject
    bookmarksopen=true,
    linktocpage=true,
    urlcolor=linkcolor, % Color of URLs
    citecolor=linkcolor, % Color of citations
    linkcolor=linkcolor, % Color of links to other pages/figures
    backref=page,
    pdfpagelabels=true,
    plainpages=false,
    colorlinks=true, % Turn off all coloring by changing this to false
    bookmarks=true,
    pdfview=FitB]{hyperref}

\usepackage[stretch=10]{microtype} % Slightly tweak font spacing for aesthetics

\usepackage[framed,numbered,autolinebreaks,useliterate]{mcode}
\usepackage{todonotes}

% This package is for plotting graphs
\usepackage{pgfplots}

%\setlength\parindent{0pt} % Uncomment to remove all indentation from paragraphs

%----------------------------------------------------------------------------------------
%    DEFINITION OF EXPERIMENTS
%----------------------------------------------------------------------------------------

% Template: \newexperiment{<abbrev>}[<short form>]{<long form>}
% <abbrev> is the reference to use later in the .tex file in \experiment{}, the <short form> is only used in the table of contents and running title - it is optional, <long form> is what is printed in the lab book itself

\newexperiment{example}[Example experiment]{This is an example experiment}
\newexperiment{example2}[Example experiment 2]{This is another example experiment}
\newexperiment{example3}[Example experiment 3]{This is yet another example experiment}

\newsubexperiment{subexp_example}[Example sub-experiment]{This is an example sub-experiment}
\newsubexperiment{subexp_example2}[Example sub-experiment 2]{This is another example sub-experiment}
\newsubexperiment{subexp_example3}[Example sub-experiment 3]{This is yet another example sub-experiment}

%----------------------------------------------------------------------------------------
\newcommand{\HRule}{\rule{\linewidth}{0.5mm}} % Command to make the lines in the title page

\setlength\parindent{0pt} % Removes all indentation from paragraphs

\begin{document}

%----------------------------------------------------------------------------------------
%    TITLE PAGE
%----------------------------------------------------------------------------------------
%\frontmatter % Use Roman numerals for page numbers

%\begin{center}

%

\title{
\begin{center}
\href{http://www.bradley.edu}{\includegraphics[height=0.5in]{figs/logoBU1-Print}}
\vskip10pt
\HRule \\[0.4cm]
{\Huge \bfseries Laboratory Notebook \\[0.5cm] \Large Mobile Cart}\\[0.4cm] % Degree
\HRule \\[1.5cm]
\end{center}
}
\author{Jason Braker \\ \\\Large jbraker@mail.bradley.edu} % Your name and email address
\date{Beginning November 23, 2020} % Beginning date
\maketitle

%\maketitle % Title page

\printindex
\tableofcontents % Table of contents
\newpage % Start lab look on a new page

\begin{addmargin}[0cm]{0cm} % Makes the text width much shorter for a compact look

\pagestyle{scrheadings} % Begin using headers

%----------------------------------------------------------------------------------------
%    LAB BOOK CONTENTS
%----------------------------------------------------------------------------------------

\labday{Monday, November 23, 2020}
\experiment{Initial Setup}

Today I met with Dr. Miah to discuss the project. We plan to have a mobile robot, a moving remote, and some fixed RF sensors. The idea is to use the EKF-SLAM algorithm to determine the location of the fixed RF sensors as well as the location of the remote. The robot should then use machine learning techniques to determine the linear and angular velocities needed to follow the remote.

\vspace{12pt}
I set up a GitHub repository to contain the project materials using the senior project repo as a template. I then invited Dr. Miah as a collaborator in the repo.

\experiment{Matlab Simulation}
I started copying code from the EKF-SLAM example and modifying it to have a moving remote instead of fixed waypoints.
%
%-------------------------------------------

\labday{Tuesday, November 24, 2020}
\experiment{Matlab Simulation}

Today I worked some more on the Matlab simulation. I changed the code to use a differential drive robot instead of a car-like robot. I got the simulation running, but the position of the remote was not being estimated correctly.

\vspace{12pt}
After some debugging, I discovered that the coordinates of the remote were in the robot's local frame, but were used as if they were in the global frame. I updated the coordinate calculation to find global coordinates, and the robot was able to estimate the position of the remote.

\vspace{12pt}
I noticed that the true position of the robot was being used to compute the control inputs $v$ and $\omega$. Since the robot does not know its true position, I decided that it should be using the estimated position instead. When I did that, the robot got quite far off on its estimate of the position, but the actual robot still generally followed the remote. I thought maybe getting observations more frequently would help the estimate be more accurate. Since the time between observations was set at 8 seconds for a single sensor with a reflector, I decided to change it to 2 seconds since using four reflectors and sensors would require only 1/4 of the rotation needed to get the full measurement. When I used 2 seconds, the robot was able to accurately estimate its position and follow the remote.

\experiment{CoppeliaSim Simulation}
I also worked on putting together a CoppeliaSim simulation. I placed five beacons in a 10 [m] by 10 [m] region and defined the trajectory of the remote to follow the square with corners (2.5, 2.5), (-2.5, 2.5), (-2.5, -2.5), and (2.5, -2.5). Then I copied the code from the Matlab simulation and updated it to integrate into CoppeliaSim. Although the generated figure looks intriguing, I do not think I would like to have a shopping cart that follows me the way this one does. See \autoref{fig:incorrectCoppSim} for the results of the simulation as it is now. I will have to look more into this next time.

\begin{figure}[h!]
    \center
    \includegraphics[width=4in]{figs/img/incorrectCoppeliaSimSimulation.png}
    \caption{Current Result of CoppeliaSim Simulation}
    \label{fig:incorrectCoppSim}
\end{figure}

%-------------------------------------------

\labday{Tuesday, December 22, 2020}
\experiment{CoppeliaSim Simulation}
%
Today I worked some more on the CoppeliaSim simulation. I tried to figure out why the robot is not behaving as intended, but was unable to locate the issue.

%-------------------------------------------

\labday{Monday, December 28, 2020}
\experiment{Reflector Design}
%
Today I worked on the design of the parabolic reflector array that will be used on top of the robot. I drew a 3d model of a paraboloidal reflector with brackets to hold the XBee modules in place, and a press-fit joint to the stepper motor. I 3d printed some small test models to verify that the XBee mounting mechanism worked properly and the press-fit joint fit well on the stepper motor. I will try to print the reflector array overnight as it will take a while to complete.

\vspace{12pt}
I also worked on a reflector array that is paraboloidal on the bottom and parabolic on the top. This design is not finished yet. I plan to finish it tomorrow. The finished paraboloidal reflector design is shown in \autoref{fig:paraboloidalReflector}. Also, close-ups of the XBee mounting mechanism and the stepper motor joint are shown in Figures \ref{fig:XBeeMountingBracket} and \ref{fig:stepperMotorPressFitJoint}, respectively.

\begin{figure}[h!]
    \center
    \includegraphics[width=4in]{figs/img/paraboloidalReflector.png}
    \caption{Finished Paraboloidal Reflector}
    \label{fig:paraboloidalReflector}
\end{figure}

\begin{figure}[h!]
    \center
    \includegraphics[width=3in]{figs/img/XBeeMountingBracket.png}
    \caption{XBee Mounting Mechanism}
    \label{fig:XBeeMountingBracket}
\end{figure}

\begin{figure}[h!]
    \center
    \includegraphics[width=3in]{figs/img/stepperMotorPressFitJoint.png}
    \caption{Stepper Motor Joint}
    \label{fig:stepperMotorPressFitJoint}
\end{figure}

%-------------------------------------------

\labday{Tuesday, December 29, 2020}
\experiment{Reflector Design}
%
Today I worked on the parabolic/paraboloidal reflector design. The finished design is shown in \autoref{fig:parabolicParaboloidalReflector}. This design uses the same XBee mounting mechanism and stepper motor joint as the purely paraboloidal design.

\begin{figure}[h!]
    \center
    \includegraphics[width=4in]{figs/img/parabolicParaboloidalReflector.png}
    \caption{Finished Parabolic/Paraboloidal Reflector}
    \label{fig:parabolicParaboloidalReflector}
\end{figure}

After finishing the parabolic/paraboloidal design, I worked on a bracket to mount the stepper motor to the BudgetBot chassis. Although there are some screw holes in the BudgetBot chassis, they were not in convenient locations, and I did not want to drill new ones at this time. Therefore, I decided to use the screws that hold the top and bottom plates together in the back. The finished stepper motor bracket is shown in \autoref{fig:stepperMotorBracket}

\begin{figure}[h!]
    \center
    \includegraphics[width=4in]{figs/img/stepperMotorBracket.png}
    \caption{Finished Stepper Motor Bracket}
    \label{fig:stepperMotorBracket}
\end{figure}

While waiting for the paraboloidal reflector array to finish printing, I tried to get the XBee modules up and running. I installed XCTU from the Digi website. I put one of the XBee modules into the USB adapter and plugged it into the computer. I was able to add it into XCTU, but when I clicked on it to try to configure it, I got a firmware error. First, I tried downloading the new firmware package to the computer, but that did not solve the problem. Then I tried the legacy firmware package. It would download for a while, but I kept getting an error at some point in the process that would stop it, so I would have to restart the process. My internet connection is not very good, so I might be able to accomplish this when I have a better connection.

\vspace{12pt}
When the paraboloidal reflector array finished printing, I cleaned all of the support material off of it and prepared it for use. I lined the inside of the reflectors with foil tape to make them reflective. Then I installed the XBee modules. I did not connect any wires to the XBees, though. An image of the assembled robot is shown in \autoref{fig:assembledRobot}.

\begin{figure}[h!]
    \center
    \includegraphics[width=5in]{figs/img/robotImages/assembledRobot.jpg}
    \caption{Assembled Robot}
    \label{fig:assembledRobot}
\end{figure}

\end{addmargin}

\end{document}


%%% Local Variables:
%%% mode: latex
%%% TeX-master: t
%%% End: